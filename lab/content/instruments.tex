\section{Приборы, используемые в работе}

Это опциональный раздел отчета, который, тем не менее, требуется при сдаче ряда дисциплин.
В нем могут быть таблицы, рисунки и прочая лабуда, которую надо оформлять в соответствии СТП–01–2010.

\paragraph{Пример оформления таблицы} В стандарте нет четкого указания насчет допустимой ширины таблиц, но во всех примерах они почему-то 
занимают всю ширину страницы:

\begin{table}[h!]
  \caption{Приборы, используемые в работе}
  \renewcommand{\tabcolsep}{0.7em}
  \begin{tabularx}{\textwidth}{| c | X | c | c | X |} % Большими буквами!
  \hline
   № &  Наименование & Тип & Заводской номер & Основные технические характеристики \\ \hline

   1 & Генератор сигналов низкочастотный & Г4-117 & &
   Диапазон генерируемых частот: 20 Гц $ ... $ 10 МГц \\ \hline

   2 & Осциллограф универсальный двухканальный & C1-117 & &
   Предел измерений: 10 МГц \par
   Погрешность коэффициента: $ \pm4 \% $ \\ \hline

   3 & Генератор импульсов & В5-54 & &
   Предел измерений: 50 В \\ \hline  
  \end{tabularx}
\end{table}

\textbf{На все рисунки, таблицы и листинги должны быть ссылки в тексте отчета.}

\newpage